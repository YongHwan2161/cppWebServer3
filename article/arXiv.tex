% !TEX TS-program = xelatex
%
%====================================================
% Channel Graph Theory - Final Version
%====================================================
\documentclass[12pt, a4paper]{article}

\usepackage[a4paper, left=3cm, right=3cm, top=2.5cm, bottom=2.5cm]{geometry}
\usepackage{fontspec}    % for xelatex
\usepackage{xunicode}    % unicode support
\usepackage{xltxtra}     % extra customizations for XeLaTeX
\usepackage{amsmath, amssymb}
\usepackage{graphicx}
\usepackage{fancyhdr}
\usepackage{tabularx}
\usepackage{multirow}
\usepackage{calc}
\usepackage[hyphens]{url}
\usepackage[hidelinks]{hyperref}
\usepackage{adjustbox}
\usepackage{microtype}
\usepackage{titlesec}
\usepackage{setspace}
\usepackage{doclicense}

% If you want to use Korean properly with XeLaTeX,
% you may need to load additional packages or set
% a Korean font, e.g.:
% \usepackage{kotex}
% \setmainhangulfont{NanumMyeongjo}

%====================================================
% Custom Font / Layout Settings
%====================================================
% change heading font (optional):
\newfontfamily\headingfont[Color=000000]{Arial}

% line spacing:
\linespread{1.2}

% Define "volume" for the top info row:
\newcommand{\volume}{Vol.\,1}

%====================================================
% Running Heads
%====================================================
\fancyhf{} % clear all header and footer settings
\fancyhead[LO]{Yonghwan Kim}
\fancyhead[RE]{Channel Graph Theory}
\fancyhead[RO, LE]{\thepage}
\pagestyle{fancy}

%====================================================
% Corresponding Author
%====================================================
\newcommand{\correspondingauthor}[1]{Yonghwan Kim, My University, \href{mailto:ant713800@gamil.com}{ant713800@gamil.com}}

%====================================================
% Hyphenation rules (optional)
%====================================================
\hyphenation{
    hy-phe-na-tion
    num-bered
}

\raggedbottom

%====================================================
% BEGIN DOCUMENT
%====================================================

\begin{document}
\thispagestyle{plain}

%====================================================
% First page header (DOI + license info)
%====================================================
\newcolumntype{Y}{>{\raggedleft\arraybackslash}X}
\noindent
\begin{tabularx}{\textwidth}{XY}
\multirow{2}{*}{\doclicenseImage[imagewidth=0.5\linewidth]} & \small{\textit{Constructions} \volume \  (\the\year)} \\
& \small{doi{10.24338/cons-xxx}} \\
\end{tabularx}

\vspace{30pt}
\begin{center}
    \LARGE\headingfont{\textbf{Channel Graph Theory: A Novel Approach to Graph-Based Database Management}}
\end{center}

\begin{center}
\vspace{4pt}
\large
    Yonghwan Kim\textsuperscript{1}
    
\small
   \textsuperscript{1} My University

\end{center}

\vspace{6pt}
\begin{small}
\begin{center}
\textbf{Abstract}
\end{center}

\begin{adjustwidth}{20pt}{20pt}
\small 
This paper introduces \textit{Channel Graph Theory}, a novel approach to graph database management that incorporates the concept of \textbf{channels} as independent connection pathways within vertices. In contrast to traditional graph databases—where edges directly link vertices without additional internal structure—Channel Graph Theory enriches each vertex with multiple, concurrently accessible channels. These allow vertices to hold several distinct sets of relationships, enabling advanced modeling of complex networks. We present theoretical foundations, detail an implementation strategy emphasizing axis-based link management, and discuss design choices for efficient memory allocation and data persistence. The performance analysis reveals that the channel-based approach can streamline graph operations, particularly in scenarios requiring multi-faceted or multilingual text representation, temporal tracking, and robust cycle management.
\end{adjustwidth}

\end{small}

%====================================================
% Introduction
%====================================================
\vspace{10pt}
\section{Introduction}\label{Sec:Introduction}
Graph databases play a vital role in managing complex data structures interconnected through a variety of relationships. Traditional systems often rely on straightforward vertex-edge models, which may struggle when multiple disjoint or specialized link patterns exist between vertices. To address these limitations, we propose \emph{Channel Graph Theory}, an approach where each vertex can maintain multiple \emph{channels}—independent pathways that each represent a distinct set of relationships.

We show how this architecture accommodates advanced modeling scenarios. In particular, Channel Graph Theory proves advantageous for string processing and memory management, two domains where large and diverse data relationships push conventional models to their limits.

%====================================================
% Channel Graph Theory
%====================================================
\section{Channel Graph Theory}\label{Sec:Theory}
\subsection{Core Concepts}
Channel Graph Theory is built upon three fundamental concepts:
\begin{itemize}
    \item \textbf{Vertices}: The basic units of data storage, each holding a unique index in the range 0--255.
    \item \textbf{Channels}: Independent connection pathways within vertices, enabling multiple sets of relationships without overlapping or interfering with one another.
    \item \textbf{Axes}: Connection properties that define the characteristics of links between channels, facilitating flexible and organized link categorization.
\end{itemize}

\subsection{string Structure}
A key application of the channel-based model is text and string management:
\begin{itemize}
    \item \textbf{Token Management}: Splits text data into tokens and assigns them to vertices for efficient lookup.
    \item \textbf{Cycle Formation}: Uses cycles of token vertices to represent discrete strings or logical groupings, preserving semantic context.
    \item \textbf{UTF-8 Support}: Ensures robust handling of multilingual data, including multi-byte characters such as CJK (Chinese, Japanese, Korean).
\end{itemize}

%====================================================
% Cycle Management
%====================================================
\section{Cycle Management}\label{Sec:Cycles}
\subsection{Cycle Detection}
Cycles are integral to maintaining string boundaries or thematic loops:
\begin{itemize}
    \item \textbf{Path Tracking}: Employs a visited-vertex list for efficient detection of cycle closure.
    \item \textbf{Maximum Cycle Length}: Implements a fixed threshold to avoid unbounded or unintended loops.
    \item \textbf{Memory-efficient Implementation}: Uses lightweight structures to track visited paths, reducing overhead.
\end{itemize}

\subsection{Path Operations}
While preserving cycle integrity, two operations are paramount:
\begin{itemize}
    \item \textbf{Path Insertion}: Introduces new vertices into an existing cycle, requiring careful link updates.
    \item \textbf{Path Deletion}: Removes node sequences while preserving the overall cycle continuity.
\end{itemize}

%====================================================
% string Processing
%====================================================
\section{string Processing}\label{Sec:strings}
\subsection{Token Optimization}
Channel-based storage allows intelligent token optimization:
\begin{itemize}
    \item \textbf{Combination Detection}: Merges frequently co-occurring tokens into compact forms.
    \item \textbf{Pattern Matching}: Uses channel-based indexing to speed up repeated phrase detection.
    \item \textbf{Storage Efficiency}: Reuses tokens when possible, minimizing duplication.
\end{itemize}

\subsection{Channel Management}
Channel handling is tuned for multi-channel string structures:
\begin{equation}
    \mathrm{Channel}_{\mathrm{new}} = 
    \begin{cases}
        \mathrm{Channel}_{\mathrm{prev}} + 1 & \text{if token is repeated}\\
        \mathrm{recycle\_or\_create()} & \text{otherwise}
    \end{cases}
\end{equation}

\subsection{UTF-8 Processing}
Robust UTF-8 support ensures:
\begin{itemize}
    \item \textbf{Multi-byte Handling}: Correctly interprets 1--4 byte encodings.
    \item \textbf{Invalid Sequence Detection}: Flags and isolates malformed byte sequences.
    \item \textbf{Escape Sequence Generation}: Provides safe encodings for storage and transfer.
\end{itemize}

%====================================================
% Memory Management
%====================================================
\section{Memory Management}\label{Sec:Memory}
\subsection{\texttt{node} Structure}
Each node (vertex) follows a structured memory layout:
\begin{itemize}
    \item \textbf{Allocated Size Power (2 bytes)}: Expresses the node's total size as a power of two for easy block alignment.
    \item \textbf{Actual Used Size (4 bytes)}: Tracks the amount of memory currently utilized.
    \item \textbf{Channel Count (2 bytes)}: Indicates how many channels are currently active in the node.
    \item \textbf{Channel Offsets Table}: Lists the start addresses of each channel (4 bytes per channel offset).
\end{itemize}

\subsection{Free Space Management}
Efficient free space management is essential:
\begin{equation}
    \mathrm{FreeBlock} = \{ \mathrm{size}: \mathrm{uint},\, \mathrm{offset}: \mathrm{long} \}
\end{equation}
Key strategies include:
\begin{itemize}
    \item \textbf{Power-of-2 Sizing}: Reduces external fragmentation and accelerates allocations.
    \item \textbf{Block Tracking and Reuse}: Maintains a global registry of free blocks to enable rapid finds.
    \item \textbf{Dynamic Space Allocation}: Allows resizing or relocating nodes while maintaining data consistency.
\end{itemize}

%====================================================
% Link Management
%====================================================
\section{Link Management}\label{Sec:Links}
\subsection{Axis-Based Connection System}
The system implements four distinct axis types for specialized link management:
\begin{itemize}
    \item \textbf{Token Search Axis (0)}: Implements a binary tree structure for efficient token data retrieval. Each axis maintains exactly two link entries, enabling logarithmic-time search operations through the token hierarchy.
    
    \item \textbf{Token Data Axis (1)}: Facilitates token separation and lookup in data arrays. This axis optimizes the reuse of existing tokens and manages efficient data partitioning.
    
    \item \textbf{String Axis (2)}: Creates cyclic structures connecting token vertices to represent longer strings. This enables efficient storage and sequential access to string data while maintaining semantic relationships.
    
    \item \textbf{Property Axis (3)}: Defines vertex characteristics through node index references. Using channel 0 as the default, this axis provides an extensible system for attribute definition, with node 0 typically indicating string start points.
\end{itemize}

\subsection{Axis Implementation}
The axis system follows a structured memory layout:
\begin{equation}
    \mathrm{AxisEntry} = \{\mathrm{number}: \mathrm{uint16},\, \mathrm{offset}: \mathrm{uint32}\}
\end{equation}

Key features include:
\begin{itemize}
    \item \textbf{Binary Tree Structure}: Token search axis maintains a strict binary organization for efficient traversal.
    \item \textbf{Cyclic Management}: String axis enables cycle creation and maintenance for string representation.
    \item \textbf{Property Extension}: Flexible property system allowing arbitrary attribute definition through node indexing.
\end{itemize}

\subsection{Memory Organization}
Axis data is organized hierarchically:
\begin{verbatim}
[Axis Count(2)] [Axis Table(6*N)] [Axis Data]
  where N = number of axes
\end{verbatim}

Each axis entry occupies 6 bytes:
\begin{itemize}
    \item 2 bytes: Axis number
    \item 4 bytes: Offset to axis data
\end{itemize}

%====================================================
% Core Memory Management
%====================================================
\section{Core Memory Management}\label{Sec:CoreMemory}
The system utilizes a \texttt{CoreMap} structure for top-level node management:
\begin{equation}
    \mathrm{CoreMap} = \{\mathrm{core\_position}: \mathrm{int},\, \mathrm{is\_loaded}: \mathrm{int},\, \mathrm{file\_offset}: \mathrm{long}\}
\end{equation}
This design offers:
\begin{itemize}
    \item \textbf{Dynamic Loading/Unloading}: Automatically pages nodes in and out based on usage.
    \item \textbf{Efficient Memory Utilization}: Avoids permanently loading infrequently used nodes.
    \item \textbf{Fast Access}: Maintains quick references for frequently accessed nodes.
\end{itemize}

%====================================================
% Performance Analysis
%====================================================
\section{Performance Analysis}\label{Sec:Performance}
\subsection{Memory Efficiency}
Our implementation demonstrates strong memory efficiency:
\begin{itemize}
    \item \textbf{Optimized Structures}: Minimizes overhead via channel-based organization.
    \item \textbf{Free Space Reuse}: Dynamically recycles memory blocks to reduce fragmentation.
    \item \textbf{Sparse Loading}: Loads only necessary nodes on demand.
\end{itemize}

\subsection{Operation Complexity}
Common operations show favorable complexity:
\begin{itemize}
    \item \textbf{Node Creation}: $O(1)$ using pre-allocated structures and power-of-2 block management.
    \item \textbf{Link Creation}: $O(\log n)$, primarily due to searching and updating free space structures.
    \item \textbf{Channel Addition}: $O(1)$, as channels are appended in a known offset table.
    \item \textbf{Axis Management}: $O(1)$, leveraging direct index references.
\end{itemize}

%====================================================
% Use Cases
%====================================================
\section{Use Cases}\label{Sec:UseCases}
\begin{itemize}
    \item \textbf{Multi-relationship Modeling}: Social networks or knowledge graphs with overlapping user-group relationships.
    \item \textbf{Temporal Data Tracking}: Historical records or versioned documents where time-based links are crucial.
    \item \textbf{Bidirectional Searching}: Applications like route planning or genealogical data exploration requiring forward and backward tracing.
    \item \textbf{Complex Text Data}: string parsing with multilingual tokenization and cycle-based representation.
\end{itemize}

%====================================================
% Future Work
%====================================================
\section{Future Work}\label{Sec:Future}
Potential avenues for further research:
\begin{itemize}
    \item \textbf{Distributed Implementation}: Scaling Channel Graph Theory across multiple nodes or data centers.
    \item \textbf{Advanced Caching Strategies}: Leveraging predictive models to prefetch high-demand nodes.
    \item \textbf{Compression Techniques}: Reducing the footprint of large datasets without compromising performance.
    \item \textbf{Query Optimization}: Developing a specialized language and engine for channel-based queries.
\end{itemize}

%====================================================
% Conclusion
%====================================================
\section{Conclusion}\label{Sec:Conclusion}
Channel Graph Theory adds a layer of flexibility and scalability to traditional graph database concepts by allowing each vertex to host multiple, independent channels. This architecture simplifies the management of complex or overlapping relationships, significantly benefits text-based applications such as multilingual string processing, and enhances cycle detection and memory reuse strategies. Our initial implementation demonstrates efficient handling of both small-scale and large-scale datasets, underscoring the viability of Channel Graph Theory as a powerful, general-purpose solution for modern graph data challenges.

%====================================================
% Conflict of interest statement & Data availability
%====================================================
\section*{Conflict of interest statement}
The author declares no competing interests.

\section*{Data availability statement}
Implementation code and test data are available at:
\begin{center}
\url{https://github.com/YourRepository/ChannelGraphTheory}
\end{center}

%====================================================
% Bibliography
%====================================================
\bibliography{bibliography}

%====================================================
% END DOCUMENT
%====================================================
\end{document}
